% Copyright 2018 by Kpym
%
% This file may be distributed and/or modified
%
% 1. under the LaTeX Project Public License and/or
% 2. under the GNU Public License.
%
% ===================================================================
%
% Available transforms :
%
% ---------------------------
% translation=A--B
%   is the translation that with vector AB (<=> sends A to B)
% ---------------------------
% rotation=A--O--B
%   is the rotation around O that sens [O,A) to [O,B) (<=> sends O to O and A to B if OA=OB)
% ---------------------------
% homothety=A--O--B
%   is the homothety with center O and ratio OB:OA (<=> sends O to O and A to B if A,O,B aligned)
% ---------------------------
% central symmetry=A
%   is the symmetry with center A (<=> sends A to A)
% ---------------------------
% projection=A--B
%   is the orthogonal projection to (AB) (<=> sends A to A and B to B)
% ---------------------------
% slanted projection=A--B--C
%   is the projection to (AB) with direction (BC) (<=> sends A to A,B to B and C to B)
% ---------------------------
% symmetry=A--B
%   is the orthogonal symmetry to (AB) (<=> sends A to A and B to B)
% ---------------------------
% slanted symmetry=A--B--C
%   is the symmetry with axis (AB) and direction (BC) (<=> sends A to A, B to B and C to (B!-1!C))
%
% ===================================================================
%
% How to use it (by examples) :
%
% you can add this transformations as styles anywhere you can use for example [rotate=20]
%
% ([central symmetry=A]B) coordinate(C), will construct C such that A is the middle of BC
% ([rotation=A--O--B]C) coordinate(D), if C is on (OA) will construct D on OB such that OC=OD
%
% ===================================================================



\typeout{tikz-euclid.transforms v1.0}

% ===================================================================
% Some tools
% ===================================================================


% Set (\pgf@x#1,\pgf@y#1) := (#2)
% ------------------------------
\def\etr@get#1#2{
  \pgf@process{\pgfpointanchor{#2}{center}}%
  \csname pgf@x#1\endcsname=\pgf@x%
  \csname pgf@y#1\endcsname=\pgf@y%
}
% Subtract (\pgf@x#1,\pgf@y#1) from (\pgf@x#2,\pgf@y#2)
% ------------------------------
\def\etr@vec#1#2{
  \advance\csname pgf@x#2\endcsname by-\csname pgf@x#1\endcsname%
  \advance\csname pgf@y#2\endcsname by-\csname pgf@y#1\endcsname%
}
% and normalize (\pgf@x#1,\pgf@y#1) into (\pgf@x#2,\pgf@y#2)
% ------------------------------
\def\etr@nrmlz#1to#2{
  \pgfmathatantwo{\the\csname pgf@y#1\endcsname}{\the\csname pgf@x#1\endcsname}%
  \let\pgf@tmp=\pgfmathresult%
  \pgfmathcos@{\pgf@tmp}\csname pgf@x#2\endcsname=\pgfmathresult pt\relax%
  \pgfmathsin@{\pgf@tmp}\csname pgf@y#2\endcsname=\pgfmathresult pt\relax%
}
% Subtract (\pgf@x#1,\pgf@y#1) from (\pgf@x#2,\pgf@y#2)
% and normalize (\pgf@x#2,\pgf@y#2)
% ------------------------------
\def\etr@nvec#1#2{
  \etr@vec{#1}{#2}
  \etr@nrmlz{#2}to{#2}
}

% ===================================================================
% The main transforms as styles
% ===================================================================

\tikzset{
  translation/.code args={#1--#2}{
    % get A--B
    \etr@get{a}{#1}\etr@get{b}{#2}
    % the directions AB
    \etr@vec{a}{b}
    % the translation transform
    \pgfkeysalso{shift={(\pgf@xb,\pgf@yb)}}%
  },
  rotation/.code args={#1--#2--#3}{
    % get A--B--C
    \etr@get{a}{#1}\etr@get{b}{#2}\etr@get{c}{#3}
    % the directions BA and BC
    \etr@vec{b}{a}\etr@vec{b}{c}
    % get the arguments of BA and BC
    \pgfmathatantwo{\the\pgf@ya}{\the\pgf@xa}%
    \let\pgf@anglea=\pgfmathresult%
    \pgfmathatantwo{\the\pgf@yc}{\the\pgf@xc}%
    \let\pgf@anglec=\pgfmathresult%
    % the homothety transform
    \pgfkeysalso{rotate around={\pgf@anglec-\pgf@anglea:(#2)}}
  },
  homothety/.code args={#1--#2--#3}{
    % get A--B--C
    \etr@get{a}{#1}\etr@get{b}{#2}\etr@get{c}{#3}
    % the directions BA and BC
    \etr@vec{b}{a}\etr@vec{b}{c}
    % normalised vector of BA is (\pgf@xb,\pgf@yb)
    \etr@nrmlz{a}to{b}
    % calculate the homothety ratio
    \pgfmathsetmacro{\pgf@lambda}{(\the\pgf@xc*\the\pgf@xb+\the\pgf@yc*\the\pgf@yb)/veclen(\the\pgf@xa,\the\pgf@ya)}
    % the homothety transform
    \pgfkeysalso{scale around={\pgf@lambda:(#2)}}
  },
  central symmetry/.style={
    scale around={-1:(#1)}
  },
  projection/.code args={#1--#2}{
    % get A--B
    \etr@get{a}{#1}\etr@get{b}{#2}
    % the normalized directions AB
    \etr@nvec{a}{b}
    % the projection transform
    \pgfkeysalso{cm={\pgf@xb*\pgf@xb,\pgf@xb*\pgf@yb,\pgf@xb*\pgf@yb,\pgf@yb*\pgf@yb,(\pgf@xa,\pgf@ya)},shift={(-\pgf@xa,-\pgf@ya)}}%
  },
  slanted projection/.code args={#1--#2--#3}{
    % get A--B--C
    \etr@get{a}{#1}\etr@get{b}{#2}\etr@get{c}{#3}
    % the normalized directions BA and BC
    \etr@nvec{b}{a}\etr@nvec{b}{c}
    % calculate the matrix coefficients
    \pgfmathsetmacro{\pgf@det}{\the\pgf@xa*\the\pgf@yc-\the\pgf@ya*\the\pgf@xc}
    \pgfmathsetmacro{\pgf@tmpx}{-\the\pgf@xc/\pgf@det}
    \pgfmathsetmacro{\pgf@tmpy}{\the\pgf@yc/\pgf@det}
    % the projection transform
    \pgfkeysalso{cm={\pgf@tmpy*\the\pgf@xa,\pgf@tmpy*\the\pgf@ya,\pgf@tmpx*\the\pgf@xa,\pgf@tmpx*\the\pgf@ya,(\pgf@xb,\pgf@yb)},shift={(-\pgf@xb,-\pgf@yb)}}%
  },
  symmetry/.code args={#1--#2}{
    % get A--B
    \etr@get{a}{#1}\etr@get{b}{#2}
    % the normalized directions AB
    \etr@nvec{a}{b}
    % the symmetry transform
    \pgfkeysalso{cm={2*\pgf@xb*\pgf@xb-1,2*\pgf@xb*\pgf@yb,2*\pgf@xb*\pgf@yb,2*\pgf@yb*\pgf@yb-1,(\pgf@xa,\pgf@ya)},shift={(-\pgf@xa,-\pgf@ya)}}%
  },
  slanted symmetry/.code args={#1--#2--#3}{
    % get A--B--C
    \etr@get{a}{#1}\etr@get{b}{#2}\etr@get{c}{#3}
    % the normalized directions BA and BC
    \etr@nvec{b}{a}\etr@nvec{b}{c}
    % calculate the matrix coefficients
    \pgfmathsetmacro{\pgf@det}{\the\pgf@xa*\the\pgf@yc-\the\pgf@ya*\the\pgf@xc}
    \pgfmathsetmacro{\pgf@tmpx}{-\the\pgf@xc/\pgf@det}
    \pgfmathsetmacro{\pgf@tmpy}{\the\pgf@yc/\pgf@det}
    % the projection transform
    \pgfkeysalso{cm={2*\pgf@tmpy*\the\pgf@xa-1,2*\pgf@tmpy*\the\pgf@ya,2*\pgf@tmpx*\the\pgf@xa,2*\pgf@tmpx*\the\pgf@ya-1,(\pgf@xb,\pgf@yb)},shift={(-\pgf@xb,-\pgf@yb)}}%
  }
}
