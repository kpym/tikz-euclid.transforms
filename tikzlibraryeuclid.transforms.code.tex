% Copyright 2018 by Kpym
%
% This file may be distributed and/or modified
%
% 1. under the LaTeX Project Public License and/or
% 2. under the GNU Public License.
%
% ===================================================================
%
% Available transforms :
%
% ---------------------------
% translation=A--B
%   is the translation with vector AB (<=> sends A to B)
% ---------------------------
% rotation=A--O--B
%   is the rotation around O that sens [O,A) to [O,B) (<=> sends O to O and A to B if OA=OB)
% ---------------------------
% homothety=A--O--B
%   it is the homothety with ratio OB:OA (<=> sends O to O, A to B iff [O,A)=[O,B))
% ---------------------------
% similarity=A--O--B
%   is the composition of <rotation=A--O--B> and <homothety=A--O--B>
%   it can be used as negative homothety if [O,A)=-[O,B)
% ---------------------------
% projection=A--B
%   is the orthogonal projection to (AB) (<=> sends A to A and B to B)
%   same as 'orthogonal projection=A--B'
% ---------------------------
% projection=A--B--C
%   is the projection to (AB) with direction (BC) (<=> sends A to A,B to B and C to B)
%   same as 'projection=A--B//B--C' and as 'slanted projection=A--B//B--C'
% ---------------------------
% projection=A--B//C--D
%   is the projection to (AB) with direction (CD)
%   same as 'slanted projection=A--B//C--D'
% ---------------------------
% symmetry=A
%   is the symmetry with center A (<=> sends A to A)
%   same as 'central symmetry=A'
% ---------------------------
% symmetry=A--B
%   is the orthogonal symmetry to (AB) (<=> sends A to A and B to B)
%   same as 'orthogonal symmetry=A--B'
% ---------------------------
% symmetry=A--B--C
%   is the symmetry with axis (AB) and direction (BC) (<=> sends A to A, B to B and C to (B!-1!C))
%   same as 'symmetry=A--B//B--C' and as as 'slanted symmetry=A--B//B--C'
% ---------------------------
% symmetry=A--B//C--D
%   is the symmetry with axis (AB) and direction (CD)
%   same as 'slanted symmetry=A--B//C--D'
%
% ===================================================================
%
% How to use it (by examples) :
%
% you can add this transformations as styles anywhere you can use for example [rotate=20]
%
% `([symmetry=A]B) coordinate(C)`, will construct C such that A is the middle of BC
% `([rotation=A--O--B]C) coordinate(D)`, if C is on (OA) will construct D on OB such that OC=OD
%
% ===================================================================
% INFO:
% the transform cm={a,b,c,d,(e,f)} is
%
%   [x] = [a c][x] + [e]
%   [y] = [b d][y] + [f]
%
% where a,b,c,d are numbers (without dimension) and e and f are dimensions (with units)
% ===================================================================



\typeout{tikz-euclid.transforms v1.0}

% ===================================================================
% Some tools
% ===================================================================

% Set (\pgf@x#1,\pgf@y#1) := (#2)
% ------------------------------
\def\etr@get#1#2{
  %\pgf@process{\pgfpointanchor{#2}{center}}%
  \tikz@scan@one@point\pgfutil@firstofone(#2)\relax
  \csname pgf@x#1\endcsname=\pgf@x%
  \csname pgf@y#1\endcsname=\pgf@y%
}
% Subtract (\pgf@x#1,\pgf@y#1) from (\pgf@x#2,\pgf@y#2)
% ------------------------------
\def\etr@vec#1#2{
  \advance\csname pgf@x#2\endcsname by-\csname pgf@x#1\endcsname%
  \advance\csname pgf@y#2\endcsname by-\csname pgf@y#1\endcsname%
}
% and normalize (\pgf@x#1,\pgf@y#1) into (\pgf@x#2,\pgf@y#2)
% ------------------------------
\def\etr@nrmlz#1to#2{
  \pgfmathatantwo{\the\csname pgf@y#1\endcsname}{\the\csname pgf@x#1\endcsname}%
  \let\pgf@tmp=\pgfmathresult%
  \pgfmathcos@{\pgf@tmp}\csname pgf@x#2\endcsname=\pgfmathresult pt\relax%
  \pgfmathsin@{\pgf@tmp}\csname pgf@y#2\endcsname=\pgfmathresult pt\relax%
}
% Subtract (\pgf@x#1,\pgf@y#1) from (\pgf@x#2,\pgf@y#2)
% and normalize (\pgf@x#2,\pgf@y#2)
% ------------------------------
\def\etr@nvec#1#2{
  \etr@vec{#1}{#2}
  \etr@nrmlz{#2}to{#2}
}
% Subtract (\pgf@x#1,\pgf@y#1) from (\pgf@x#2,\pgf@y#2)
% and divide by (1cm,1cm)
% ------------------------------
\def\etr@cmvec#1#2{
  \etr@vec{#1}{#2}
  \pgfmathsetlength{\csname pgf@x#2\endcsname}{.0351459999986\csname pgf@x#2\endcsname}
  \pgfmathsetlength{\csname pgf@y#2\endcsname}{.0351459999986\csname pgf@y#2\endcsname}
}
% Set the matrix transform
% ------------------------------
\newdimen\etr@dx
\newdimen\etr@dy
\def\etr@transform#1#2#3#4#5#6{
  \pgfmathsetmacro{\pgf@cma}{#1}
  \pgfmathsetmacro{\pgf@cmb}{#2}
  \pgfmathsetmacro{\pgf@cmc}{#3}
  \pgfmathsetmacro{\pgf@cmd}{#4}
  \pgfmathsetlength{\etr@dx}{#5}
  \pgfmathsetlength{\etr@dy}{#6}
  \pgftransformcm{\pgf@cma}{\pgf@cmb}{\pgf@cmc}{\pgf@cmd}{\pgfpoint{\etr@dx}{\etr@dy}}
  \pgfmathsetlength{\etr@dx}{-\etr@dx}
  \pgfmathsetlength{\etr@dy}{-\etr@dy}
  \pgftransformcm{1}{0}{0}{1}{\pgfpoint{\etr@dx}{\etr@dy}}
}
% -------------------------------------------------------------------
% Parse tools based on private conversation with jfbu
% -------------------------------------------------------------------

% ---- parse symetry ----
\def\symmetryone#1\parsesymmetry{\pgfkeysalso{central symmetry/.expanded=#1}}
\def\symmetrytwo#1--#2\parsesymmetry{\pgfkeysalso{orthogonal symmetry/.expanded=#1--#2}}
\def\symmetrythree#1--#2--#3\parsesymmetry{\pgfkeysalso{slanted symmetry/.expanded=#1--#2//#2--#3}}
\def\symmetryfour#1--#2//#3--#4\parsesymmetry{\pgfkeysalso{slanted symmetry/.expanded=#1--#2//#3--#4}}

\def\parsesymmetry#1{\parse@symmetry#1--2--1--0\parse@symmetry{#1}}

\def\parse@symmetry#1--#2--#3--#4#5\parse@symmetry#6{%
    \ifcase #4
    \expandafter\symmetryone
    \or
    \expandafter\symmetrytwo
    \or
    \expandafter\parsedeepersymmetry
    \else
    \ERROR
    \fi #6\parsesymmetry
}%

\def\parsedeepersymmetry #1--#2--#3\parsesymmetry{%
    \parse@deepersymmetry #2//\parse@deepersymmetry #1--#2--#3\parsesymmetry
}%

\def\parse@deepersymmetry #1//#2\parse@deepersymmetry{%
    \if\relax\detokenize{#2}\relax
      \expandafter\symmetrythree
    \else
      \expandafter\symmetryfour
    \fi
}%

% ---- parse projection ----
\def\projectionone#1\parseprojection{\relax}
\def\projectiontwo#1--#2\parseprojection{\pgfkeysalso{orthogonal projection/.expanded=#1--#2}}
\def\projectionthree#1--#2--#3\parseprojection{\pgfkeysalso{slanted projection/.expanded=#1--#2//#2--#3}}
\def\projectionfour#1--#2//#3--#4\parseprojection{\pgfkeysalso{slanted projection/.expanded=#1--#2//#3--#4}}

\def\parseprojection#1{\parse@projection#1--2--1--0\parse@projection{#1}}

\def\parse@projection#1--#2--#3--#4#5\parse@projection#6{%
    \ifcase #4
    \expandafter\projectionone
    \or
    \expandafter\projectiontwo
    \or
    \expandafter\parsedeeperprojection
    \else
    \ERROR
    \fi #6\parseprojection
}%

\def\parsedeeperprojection #1--#2--#3\parseprojection{%
    \parse@deeperprojection #2//\parse@deeperprojection #1--#2--#3\parseprojection
}%

\def\parse@deeperprojection #1//#2\parse@deeperprojection{%
    \if\relax\detokenize{#2}\relax
      \expandafter\projectionthree
    \else
      \expandafter\projectionfour
    \fi
}%



% ===================================================================
% The main transforms as styles
% ===================================================================

\tikzset{
  translation/.code args={#1--#2}{
    % get A--B
    \etr@get{a}{#1}\etr@get{b}{#2}
    % the directions AB
    \etr@vec{a}{b}
    % the translation transform
    \pgfkeysalso{shift={(\pgf@xb,\pgf@yb)}}%
  },
  rotation/.code args={#1--#2--#3}{
    % get A--B--C
    \etr@get{a}{#1}\etr@get{b}{#2}\etr@get{c}{#3}
    % the directions BA and BC
    \etr@vec{b}{a}\etr@vec{b}{c}
    % get the arguments of BA and BC
    \pgfmathatantwo{\the\pgf@ya}{\the\pgf@xa}%
    \let\pgf@anglea=\pgfmathresult%
    \pgfmathatantwo{\the\pgf@yc}{\the\pgf@xc}%
    \let\pgf@anglec=\pgfmathresult%
    % the rotation transform
    \pgfkeysalso{rotate around={\pgf@anglec-\pgf@anglea:(#2)}}
  },
  homothety/.code args={#1--#2--#3}{
    % get A--B--C
    \etr@get{a}{#1}\etr@get{b}{#2}\etr@get{c}{#3}
    % the directions BA and BC
    \etr@vec{b}{a}\etr@vec{b}{c}
    % calculate the homothety ratio
    \pgfmathsetmacro{\pgf@lambda}{veclen(\the\pgf@xc,\the\pgf@yc)/veclen(\the\pgf@xa,\the\pgf@ya)}
    % the homothety transform
    \pgfkeysalso{scale around={\pgf@lambda:(#2)}}
  },
  similarity/.code args={#1--#2--#3}{
    % get A--B--C
    \etr@get{a}{#1}\etr@get{b}{#2}\etr@get{c}{#3}
    % the directions BA and BC
    \etr@vec{b}{a}\etr@vec{b}{c}
    % get the arguments of BA and BC
    \pgfmathatantwo{\the\pgf@ya}{\the\pgf@xa}%
    \let\pgf@anglea=\pgfmathresult%
    \pgfmathatantwo{\the\pgf@yc}{\the\pgf@xc}%
    \let\pgf@anglec=\pgfmathresult%
    % the rotation transform
    \pgfkeysalso{rotate around={\pgf@anglec-\pgf@anglea:(#2)}}
    % calculate the homothety ratio
    \pgfmathsetmacro{\pgf@lambda}{veclen(\the\pgf@xc,\the\pgf@yc)/veclen(\the\pgf@xa,\the\pgf@ya)}
    % the homothety transform
    \pgfkeysalso{scale around={\pgf@lambda:(#2)}}
  },
  orthogonal projection/.code args={#1--#2}{
    % get A--B
    \etr@get{a}{#1}\etr@get{b}{#2}
    % the normalized directions AB
    \etr@nvec{a}{b}
    % calculate matrix coefficients
    \pgfmathsetmacro{\pgf@tmpxx}{\pgf@sys@tonumber\pgf@xb\pgf@xb}
    \pgfmathsetmacro{\pgf@tmpxy}{\pgf@sys@tonumber\pgf@xb\pgf@yb}
    \pgfmathsetmacro{\pgf@tmpyy}{\pgf@sys@tonumber\pgf@yb\pgf@yb}
    % the orthogonal projection transform
    \etr@transform{\pgf@tmpxx}{\pgf@tmpxy}{\pgf@tmpxy}{\pgf@tmpyy}{\pgf@xa}{\pgf@ya}
  },
  slanted projection/.code args={#1--#2//#3--#4}{
    % get C--D and normalize it (in c)
    \etr@get{b}{#3}\etr@get{c}{#4}\etr@nvec{b}{c}
    % get B--A and normalize it (in a, keep B in b)
    \etr@get{a}{#1}\etr@get{b}{#2}\etr@nvec{b}{a}
    % calculate the matrix coefficients
    \pgfmathsetmacro{\pgf@det}{\pgf@sys@tonumber\pgf@xa\pgf@yc-\pgf@sys@tonumber\pgf@ya\pgf@xc}
    \pgfmathsetmacro{\pgf@tmpx}{-\pgf@sys@tonumber{\pgf@xc}/\pgf@det}
    \pgfmathsetmacro{\pgf@tmpy}{\pgf@sys@tonumber{\pgf@yc}/\pgf@det}
    % the slanted projection transform
    \etr@transform{\pgf@tmpy\pgf@xa}{\pgf@tmpy\pgf@ya}{\pgf@tmpx\pgf@xa}{\pgf@tmpx\pgf@ya}{\pgf@xb}{\pgf@yb}%
  },
  projection/.code={
    \parseprojection{#1}
  },
  central symmetry/.style={
    scale around={-1:(#1)}
  },
  orthogonal symmetry/.code args={#1--#2}{
    % get A--B
    \etr@get{a}{#1}\etr@get{b}{#2}
    % the normalized directions AB
    \etr@nvec{a}{b}
    % calculate matrix coefficients
    \pgfmathsetmacro{\pgf@tmpxx}{2*\pgf@sys@tonumber\pgf@xb\pgf@xb-1}
    \pgfmathsetmacro{\pgf@tmpxy}{2*\pgf@sys@tonumber\pgf@xb\pgf@yb}
    \pgfmathsetmacro{\pgf@tmpyy}{2*\pgf@sys@tonumber\pgf@yb\pgf@yb-1}
    % the orthogonal symmetry transform
    \etr@transform{\pgf@tmpxx}{\pgf@tmpxy}{\pgf@tmpxy}{\pgf@tmpyy}{\pgf@xa}{\pgf@ya}%
  },
  slanted symmetry/.code args={#1--#2//#3--#4}{
    % get C--D and normalize it (in c)
    \etr@get{b}{#3}\etr@get{c}{#4}\etr@nvec{b}{c}
    % get B--A and normalize it (in a, keep B in b)
    \etr@get{a}{#1}\etr@get{b}{#2}\etr@nvec{b}{a}
    % calculate the matrix coefficients
    \pgfmathsetmacro{\pgf@det}{\pgf@sys@tonumber\pgf@xa\pgf@yc-\pgf@sys@tonumber\pgf@ya\pgf@xc}
    \pgfmathsetmacro{\pgf@tmpx}{-2*\pgf@sys@tonumber{\pgf@xc}/\pgf@det}
    \pgfmathsetmacro{\pgf@tmpy}{2*\pgf@sys@tonumber{\pgf@yc}/\pgf@det}
    % the slanted symmetry transform
    \etr@transform{\pgf@tmpy\pgf@xa-1}{\pgf@tmpy\pgf@ya}{\pgf@tmpx\pgf@xa}{\pgf@tmpx\pgf@ya-1}{\pgf@xb}{\pgf@yb}
  },
  symmetry/.code={
    \parsesymmetry{#1}
  },
  to base/.code args={#1--#2--#3}{
    % get C--D and normalize it (in c)
    \etr@get{a}{#1}\etr@get{b}{#2}\etr@get{c}{#3}
    % get BA and BC
    \etr@cmvec{b}{a}\etr@cmvec{b}{c}
    % to base transform
    \pgftransformcm{\pgf@sys@tonumber\pgf@xa}{\pgf@sys@tonumber\pgf@ya}{\pgf@sys@tonumber\pgf@xc}{\pgf@sys@tonumber\pgf@yc}{\pgfpoint{\pgf@xb}{\pgf@yb}}
  },
  from base/.code args={#1--#2--#3}{
    % get C--D and normalize it (in c)
    \etr@get{a}{#1}\etr@get{b}{#2}\etr@get{c}{#3}
    % get BA and BC
    \etr@cmvec{b}{a}\etr@cmvec{b}{c}
    % get the determinant
    \pgfmathsetmacro{\pgf@det}{\pgf@sys@tonumber\pgf@xa\pgf@yc-\pgf@sys@tonumber\pgf@ya\pgf@xc}
    % to base transform
    \pgfkeysalso{cm={\pgf@sys@tonumber\pgf@yc/\pgf@det,-\pgf@sys@tonumber\pgf@ya/\pgf@det,-\pgf@sys@tonumber\pgf@xc/\pgf@det,\pgf@sys@tonumber\pgf@xa/\pgf@det,(0,0)},shift={(-\pgf@xb,-\pgf@yb)}}%
  }
}
